\section{MANUSCRIPTS AND PRE-PRINTS}
\subsection{arXiv Preprints}
\cventry{arXiv 2024}{Inverse Neural Rendering for Explainable Multi-Object Tracking.}{}{}{}{
{Julian Ost$^*$, \textbf{Tanushree Banerjee$^*$}, Mario Bijelic, Felix Heide} \hfill \faLink~\href{https://light.princeton.edu/publication/inverse-rendering-tracking}{Project Page} $\mid$ \href{https://drive.google.com/file/d/1IFN0Lm1MCfnCtuBaUoYgwiQMewGH-LX3/view?usp=sharing}{Paper} $\mid$ \href{https://arxiv.org/abs/2404.12359}{arXiv:2404.12359}\\
\textit{Under review at \emph{Nature Machine Intelligence}}.
We recast 3D multi-object tracking from RGB cameras as an inverse rendering problem. Our method is not just an alternate take on tracking; it enables examining generated objects and reasoning about failure cases.}

\cventry{arXiv 2024}{LLMs are Superior Feedback Providers: Bootstrapping Reasoning for Lie Detection with Self-Generated Feedback.}{}{}{}{
\textbf{Tanushree Banerjee}, Richard Zhu, Runzhe Yang, Karthik Narasimhan \hfill
\faLink~\href{https://arxiv.org/pdf/2408.13915}{Paper} $\mid$ {\href{https://arxiv.org/submit/5812542/view}{arXiv:2408.13915}}
\\
 We investigated a bootstrapping framework that leverages LLM-generated feedback to detect deception in diplomacy games. Our approach achieved a 39\% improvement over the zero-shot baseline in lying F1 without any training.
} 

\medskip
\subsection{Undergraduate Thesis}
\cventry{2024}{Inverse Neural Rendering for Explainable 3D Perception}{Princeton University.}{}{}{
Advisor: Prof. Felix Heide
\hfill \faLink~\href{https://drive.google.com/file/d/1PMMD7ejmdjBRDj_lofK5fgKERdDiWHX1/view?usp=sharing}{Thesis Report}~$\mid$~\href{http://arks.princeton.edu/ark:/88435/dsp01dz010t427}{Abstract}\\
    This thesis explores unlocking explainable 3D perception via Inverse Neural Rendering. \emph{Part I} proposes and evaluates a novel take on 3D multi-object tracking, while \emph{Part II} proposes recasting 3D object detection as an inverse generation problem.
}